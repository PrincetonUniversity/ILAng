%%%%%%%%%%%%%%%%%%%%%%%%%%%%%%%%%%%%%%%%%%%%%%%%%%%%%%%%%%%%%%%%%%%%%%%%%%%%%%%%
%2345678901234567890123456789012345678901234567890123456789012345678901234567890
%        1         2         3         4         5         6         7         8

\documentclass[letterpaper, 11 pt]{article}  % Comment this line out
                                                          % if you need a4paper
%\documentclass[a4paper, 10pt, conference]{ieeeconf}      % Use this line for a4
                                                          % paper
\usepackage{url}
\usepackage{graphicx}
\usepackage{amsmath}
\usepackage{amssymb}
\usepackage{amsfonts}
\usepackage{proof}
\usepackage{tikz}
\usepackage[margin=1.2in]{geometry}

\title{Common Instruction-Level Abstraction Issues}
\author{}

%\date{Draft Working Document: January 10, 2017}
\date{Draft Working Document: \today}

\begin{document}
\maketitle

\providecommand{\bd}[0]{\mathbb{B}}
\providecommand{\st}[1]{\mathrm{#1}}
\providecommand{\ft}[1]{\mathtt{#1}}

\section*{Timing}

\section*{Instruction Ordering}

\section*{Interrupt}

\section*{Micro-architecture and Specification}

\section*{Concurrency}

In this scenario we consider two different design options: physical and logical multi-core (logical is simultaneous multithreading). 
\subsection*{Physical multi-core} 
In a physical multi-core design computational resources are duplicated. Modern CPUs contain 2-24 physical cores, while GPUs contain thousands of physical cores. Since our hierarchical-ILA supports concurrent execution of instructions, it can be used to model such multi-core designs. More precisely, an ILA for a multi-core design includes many identical child-ILAs. Note that while multi-core designs are based on the notion of duplicating computational resources, the way they operate is different. The most obvious example is CPU vs. GPU. In most cases, execution on different GPU cores is independent, unlike the case of a CPU. We need to decide if we add this modeling power to the ILA.

\subsection*{Logical multi-core, aka SMT, HyperThreading}
In this case, the computational resources are not duplicated. Instead, the architectural state (e.g. control and general purpose registers) are duplicated. Using this approach, if the CPU includes one core, the OS views it as two \emph{logical} cores. The management of the physical core (the actual computational resource) is done at the hardware level, i.e., by the CPU and can be captured by the ILA.

%\bibliography{refs}
%\bibliographystyle{unsrt}

\end{document}

