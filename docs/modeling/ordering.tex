%%%%%%%%%%%%%%%%%%%%%%%%%%%%%%%%%%%%%%%%%%%%%%%%%%%%%%%%%%%%%%%%%%%%%%%%%%%%%%%%
%2345678901234567890123456789012345678901234567890123456789012345678901234567890
%        1         2         3         4         5         6         7         8

\documentclass[letterpaper, 11 pt]{article}  % Comment this line out
                                                          % if you need a4paper
%\documentclass[a4paper, 10pt, conference]{ieeeconf}      % Use this line for a4
                                                          % paper
\usepackage{url}
\usepackage{graphicx}
\usepackage{amsmath}
\usepackage{amssymb}
\usepackage{amsfonts}
\usepackage{proof}
\usepackage{tikz}
\usepackage[margin=1.2in]{geometry}


\usepackage{color}
\definecolor{light-gray}{gray}{0.95}
\usepackage{listings}
\lstset{ %
language=Python,                % choose the language of the code
basicstyle=\footnotesize,       % the size of the fonts that are used for the code
numbers=left,                   % where to put the line-numbers
numberstyle=\footnotesize,      % the size of the fonts that are used for the line-numbers
stepnumber=1,                   % the step between two line-numbers. If it is 1 each line will be numbered
numbersep=5pt,                  % how far the line-numbers are from the code
backgroundcolor=\color{light-gray},  % choose the background color. You must add \usepackage{color}
showspaces=false,               % show spaces adding particular underscores
showstringspaces=false,         % underline spaces within strings
showtabs=false,                 % show tabs within strings adding particular underscores
frame=single,           % adds a frame around the code
tabsize=2,          % sets default tabsize to 2 spaces
captionpos=b,           % sets the caption-position to bottom
breaklines=true,        % sets automatic line breaking
breakatwhitespace=false,    % sets if automatic breaks should only happen at whitespace
escapeinside={\%*}{*)}          % if you want to add a comment within your code
}


\title{ILA Modeling of Ordering}
\author{}

\date{Draft Working Document: \today}

\begin{document}
\maketitle

\providecommand{\bd}[0]{\mathbb{B}}
\providecommand{\st}[1]{\mathrm{#1}}
\providecommand{\ft}[1]{\mathtt{#1}}

%%%%%%%%%%%%%%%%%%%%%%%%%%%%%%%%%%%%% Body %%%%%%%%%%%%%%%%%%%%%%%%%%%%%%%%%%%%%

Currently, in the ILA model, there can be multiple instructions or child-instructions
being ready to issue (in other words, their decode conditions are true) at the same
time. If they are issued simultaneously, there is no restrictions on their commit
orders. For the time being, the ordering should be enforced at instruction issue time, which
means modeling the order by the condition in the decode condition. Later, after
we integrate memory consistency model into our ILA framework, we will be able to enforce
the ordering at commit time as well.

\section{Instruction to Instruction Ordering}
 
The ordering of instructions should be enforced at issue stage. The following code shows an 
example processor which fetches two instructions at a time. The decode expressions for the
two instructions can be true at the same time, and there is no order of the two instructions.

\lstinputlisting{code/orderExample1.py}

In order to enforce the ordering at instruction issue time, we need to introduce extra condition
at the decode expression. By setting a flag which is set by instruction A and cleared by instruction
B, we explicitly set the issue order that instruction B can only be issued after instruction
A.

\lstinputlisting{code/orderExample2.py}

\section{Instruction to Child-Instruction Ordering}

In case that the instructions should wait for child-instructions or other instructions, it can be 
modeled similarly as described above. For micro-instructions, their internal update steps are not 
outside visible, so another way of enforcing order is to create a loop in the micro-program and the
exit condition of the loop is the condition it waits for.

In the following example, the micro-instructions are designed to wait for a certain condition \textit{CondWaitFor}.
The micro-program forms a loop which keeps executing \textit{Wait} instructions until the condition
is true. 

\lstinputlisting{code/orderExample3.py}

%\bibliography{refs}
%\bibliographystyle{unsrt}

\end{document}

